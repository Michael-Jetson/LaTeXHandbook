\newcommand{\MyCMD}{\textbf{这是我们自定义的命令}}

\chapter{命令}
\section{为什么定义命令}
对于命令,大家之前也了解过,一个  \textbackslash 加上相应的单词,就可以让编译器进行相应的操作,这就是命令,但是如果我们想对某一种类似的操作进行重复使用,那么直接进行各种命令的叠加会非常麻烦,所以我们需要定义自己命令来进行各种操作的集成,这就是我们为什么要进行定义新命令的原因。
\section{定义一个新命令}
\LaTeX{}的命令相当于其他编程语言中的函数,我们定义新命令的过程跟定义函数的过程也类似,首先我们来定义一个最简单的命令
\begin{lstlisting}[language=TeX]
\newcommand{\MyCMD}{\textbf{这是我们自定义的命令}}
\end{lstlisting}

其中,\textbackslash newcommand是标识符,告诉编译器我们要定义一个新命令,然后\textbackslash MyCMD是我们定义的新命令的标识符,类似于函数名,我们在文档中定义这个标识符之后,就可以进行使用,然后第二个花括号里面是新命令的内容,也就是我们使用这个新命令之后会进行什么操作,在这里我们定义这个指令的操作是输出加粗的文字“这是往年自定义的命令”,使用方法及效果如下
\begin{lstlisting}[language=TeX]
\MyCMD
\end{lstlisting}

\MyCMD

这样,我们就可以把各种操作合并为一个简短的命令里面,从而大大提高我们的排版效率